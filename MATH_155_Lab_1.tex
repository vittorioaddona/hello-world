\documentclass[12pt]{report}
\usepackage{amsmath,amsfonts,geometry,comment}
\geometry{margin=1in}

\begin{document}
\begin{center}
\textbf{Math 155: Lab \# 1}\\ \vspace{2mm}
\textbf{Graphical/Numerical descriptive statistics, Residuals, and Models}\\ \vspace{2mm}
\textbf{Due: Today, by the end of our class period}\\ \vspace{2mm}
\end{center}
\textbf{Instructions:} To save your work for submission, you can either use \textbf{R Markdown} to create a file, or you can copy/paste results into a word processor (e.g. Word, Pages, etc.). Please include any necessary \textbf{graphs}, \textbf{R commands}, and \textbf{answers to the questions}. \vspace{6mm}

\noindent \textbf{Upload your work to our course Moodle page using the link labeled ``Lab \# 1''}. \vspace{6mm}

\noindent \textbf{Data Description:} The file \emph{\textbf{MacClass2012.csv}} was generated by a Macalester student for her Senior Honors Project. It contains a variety of information on Macalester's graduating class of 2012. Specifically, every row in the file corresponds to a single student, and the recorded variables for each of the 477 students are:\\ \vspace{-1mm}

\noindent \textbf{StudentID}:~ An ID number assigned to each student (not their College ID number).\vspace{1mm}

\noindent \textbf{FYGPA}:~ The student's first-year grade point average. \vspace{1mm}

\noindent \textbf{Sex}:~ The student's sex, categorized as either male (M) or female (F). \vspace{1mm}

\noindent \textbf{IPEDSrace}:~ The student's Integrated Postsecondary Education Data System (IPEDS)\\\phantom{1}\hspace{2.3cm}race category: non-resident alien (NR), Hispanic (HI), Asian (AS), Black (BL),\\\phantom{1}\hspace{2.15cm} Pacific Islander (IS), White (WH), Two or more races (MR). \vspace{1mm}

\noindent \textbf{FYAid}:~ The amount of financial aid received in the first-year (0, 1-15000, 15001-24999,\\\phantom{1}\hspace{1.65cm}25000-34999, 35000-44999) \vspace{1mm}

\noindent \textbf{HSClassRank}:~ The student's high-school class rank ($1^{st}$ decile, $2^{nd}$ decile, $3^{rd}$ decile,\\\phantom{1}\hspace{2.95cm} $4^{th}$ decile, $5^{th}$ decile, $6^{th}$ decile, Top 5) \vspace{1mm}

\noindent \textbf{SATreading}:~ The student's SAT reading score. \vspace{1mm}

\noindent \textbf{SATmath}:~ The student's SAT math score. \vspace{1mm}

\noindent \textbf{SATwriting}:~ The student's SAT writing score. \vspace{1mm}

\noindent \textbf{ACT}:~ The student's American College Testing (ACT) test score. \vspace{1mm}

\noindent \textbf{Major}:~ The student's major. This was not provided to us, but ascertained via an algorithm.\\\phantom{1}\hspace{1.35cm} It only selects one major for students who are double/triple majors.\vspace{1mm}

\noindent \textbf{Division}:~ The division to which the student's major belongs.\\ \vspace{4mm}

\noindent \textbf{Reading the data into RStudio:}
\begin{itemize}
\item To read the data into R, type the following command:

\emph{MacClass2012 = read.csv( ``http://www.macalester.edu/$\sim$addona/MacClass2012.csv'')}

\item Then, if you want to view the data, type in the R Console: \emph{View(MacClass2012)}
\end{itemize}

\newpage

\begin{enumerate}
\item Make a boxplot (or histogram) of the FYGPA variable. Does it show any signs of skewness, and if so, which type?

\item According to the ``1.5 IQR'' rule of thumb, first-year GPAs below what threshold value will be considered ``potential outliers''?

\item Find the mean and median first-year GPA for these 477 students. Does a comparison of the mean and median indicate skewness. Briefly explain.

\item Find the variance of the first-year GPAs.

\noindent The variance and the standard deviation both measure spread. What is the relationship between variance and standard deviation?

\noindent What is the advantage of using the standard deviation to measure spread instead of the variance?

\item Make side-by-side boxplots of the first-year GPAs by financial aid. Does it seem like financial aid is a good explantory variable for modeling GPA? Briefly explain.

\noindent Repeat for first-year GPAs by HSClassRank. Do you think HSClassRank is a good explanatory variable for modeling GPA?

\item Using the \emph{lm} function, fit a model for FYGPA that uses no explanatory variables, and report the coefficient from this model. What quantity does this coefficient represent?

\item Make a new variable called ``Residuals'' that contains all of the 477 residuals from the model you fit for FYGPA in 6. Find the sum of squared residuals.

\item Suppose we are interested in determining the relationship between first-year GPA and: (i) SATmath , (ii) ACT. Make scatterplots to describe these two relationships.

\noindent For which explanatory variable (SATmath or ACT) does the relationship seem stronger?

\item Fit a model for FYGPA that uses SATmath as an explanatory variable and report the intercept and slope of the line of best fit.

\noindent Repeat with ACT in place of SATmath.

\item Find the sum of squared residuals for the two models in 9. Does a comparison of the sum of squared residuals agree with your answer to 8.?
% Because of missing values, these sum of squared residuals are not actually directly comparable, but if we account for the missing values, the story stays the same. It gets complicated at this stage of the course to get into the missing data points issue.

\item Your friend tells you that the comparison of the explanatory variables SATmath and ACT is trivial: The slope from the FYGPA $\sim$ SATmath model is only (roughly) 0.0004, whereas the slope from the FYGPA $\sim$ ACT model is (roughly) 0.04; ACT is clearly the better explanatory variable, with a slope that is 100 times bigger!

Is your friend's reasoning sound? Why or why not?
\end{enumerate}
\end{document}
